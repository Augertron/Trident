\section{Trident Compiler Synthesis}


The synthesis section of the Trident compiler creates ``abstract''
circuits based on a given scheduled control flow graph (CFG).

After abstract circuits are built, code is generated in the underlying
technology (so far VHDL) from the information stored in the abstract
circuits.  More specifically, when the abstract components are built,
a request is made to the Circuit class to generate HDL code.

\subsection{Main Classes}

\begin{description}
\item[GenericCircuitGenerator] base class, includes commonly used functions
\item[DesignCircuitGenerator] top-level for a Trident design
\item[DatapathCircuitGenerator] implements the datapath 
\item[RegfileGenerator] implements the register file
\item[TopLevelCircuitGenerator] creates top-level connections with the XD1
\item[InterfaceGenerator] base class of interface classes
\item[MemoryInterfaceGenerator] adds logic for connecting to memories
\item[RegisterInterfaceGenerator] adds logic for connecting to the host
\item[AddressDecodeGenerator] creates address decoders and memory selectors
\end{description}

\subsection{Description}

\subsubsection{Block and Datapath}

A block circuit is created for each hyperblock found in the CFG.  Each
block circuit contains a datapath, a register file, a state machine
and corresponding control logic. For each hyperblock, every
instruction is analyzed and logic is created to implement each
instruction. For example, if a store instruction is found, a wire
connects the data to a guarding mux which is connected to the register
file.

\subsubsection{Register File}
The register file holds temporary values that are needed between block
circuits. When any primal is loaded or stored, the register file will
be accessed.  It simply contains a register for every primal that is
used in the CFG.  The connections to the register file are created in
the DesignCircuitGenerator.

\subsubsection{Block Control}
Block control is implemented in the DesignCircuitGenerator.  The
control depends on the state machine and predicates associated with
each block.

\subsubsection{State Machines}
The StateMachine base class provides basic methods for generating a
state machine.  The GenericStateMachine class generates state machines
for normally scheduled blocks, and the ModuloStateMachine class
generates state machines for modulo scheduled blocks.  The main
difference between these state machines is that the modulo state
machine will not try to exit early, while the generic state machine
could possibly.  Both types of state machines are ``one-hot'' state
machines (meaning they have one register for each individual state).
Both state machines use the last state as the start state.

Note: In order for the state machines to work properly, the reset
signal at the top-level needs to go high for at least one cycle.

\subsubsection{Host/Design Register Interface}
Logic is needed in order to handle the host's initialization of the
registers in the Trident design.  The RegisterInterfaceGenerator class
implements this functionality.  The host sends addresses and data on
the RapidArray transport which are received at the top-level of the
Trident design.  The lowest bits of the address bus denote a
register's address.  There's one address decoder for each register
that's been declared as externally visible.  The inputs of each
address decoder are connected to the lowest bits of the address bus.
During a host register read, the register read enable signal (the
address decoder output delayed 1 cycle) is used to control a guarding
mux for the data coming from the register in the register file.  Each
externally visible register has this read enable and guarding logic at
the top level.  The output of each guarding mux is then extended to
DATA\_BUS\_WIDTH bits wide.  Those outputs are OR'ed together and put
on the o\_reg bus to be sent back to the host.

When a host writes a register to the register file, the address is
decoded as described above.  In order to get the host write enable for
a register, the done signal, status' register write enable, and
address decoder are AND'ed together.  The data coming from the host is
sliced down to its declared size.  A mux controlled by the
corresponding write enable guards the data heading towards the
register file.  The mux output is OR'ed with wires created from stores
in the block circuits.  The OR output goes straight to the
corresponding register in the register file.

The addresses of registers can be found printed out after running
Trident.  When making a ``fabric.in'' file during simulation, these
addresses must be added to the XD1's register space offset (0x4000000)
in order to have the address show up on the correct bus.  Also, when
simulating, the signal to start the hardware design is treated like a
host register write.  The start signal's address is always the largest
address printed during Trident's output.

\subsubsection{Memory Interface}
The MemoryInterfaceGenerator class implements the logic connecting the
XD1's QDR2 SRAM memories with the memory uses in the hardware design.
Also, it provides host access to the specific SRAM banks.  The QDR2
RAM is mapped by RT Client logic into the second 8 Mbytes of QDR RAM
space.  Since each SRAM has the same amount of memory, each SRAM bank
holds 2MBytes.  Therefore, the SRAMs are divided into the following
address spaces (with 64-bit addressable elements): mem0 (0x800000 to
0x9FFFFF), mem1 (0xA00000 to 0xBFFFFF), mem2 (0xC00000 to 0xDFFFFF),
and mem3 (0xE00000 to 0xFFFFFF).  The memory select bits appear on the
18th and 19th bits of the address bus.  Memory selectors (modified
address decoders), which are connected to the two memory select bits,
decide which memory is being accessed.

A host memory write sends an address and data down their respective RT
Client buses.  Then, the memory selectors decide which memory is being
accessed.  Each memory's write enable is made from the AND of the
memory select's output, the done signal, and the status bus' memory
enable.  Along with supplying each memory i's w\_n\_i signal, a
memory's write enable is used as control to two muxes. The first mux
guards data inputs to the data write bus.  The second mux guards
address inputs to the address write bus.  Both buses of each memory
have an OR gate in front in order to connect the Trident hardware
design memory writes.

A host memory read sends an address down the RT Client bus.  Again,
the memory number is decoded, and therefore a memory has been
selected.  That signal, negated, is used as memory i's r\_n\_i signal.
While the r\_n\_i is active low, the address bus holds the address
specified by the host read request.  Eight cycles after the address is
placed on the SRAM address bus, the data is returned on the SRAM
bank's data bus.  Here, the read enable signal is made by AND'ing the
status bus's memory read signal with the corresponding memory select
output.  Since those signals occur eight cycles previously, a chain of
registers is used to delay the read enable signal eight cycles.  The
output of the chain of registers is used as control for a mux guarding
the incoming data.  The mux output is OR'ed with all other related
guard muxes.  Finally, the data coming from the OR gate is sliced from
72 bits down to 64 bits -- due to the difference between QDR2 SRAM and
RT Client data widths.  The sliced data is put onto the RT Client bus.

When the design requests a memory read, no memory selectors are needed
because each array and memory access has been designated a memory
during the AnalyzeHardware pass.  First, the address is calculated
inside the datapath.  During the scheduler determined cycle, the
address flows through a mux and gets sliced from 32-bits down to
20-bits.  Next, an OR gate connects the 20-bit address and other
addresses from other blocks.  Then, another OR gate connects the
previous OR gate's output with the host's address.  The output of this
OR gate goes straight to the memory's address read port.  The read
enable goes through a similar path.  It gets OR'ed with other blocks'
read enables and also with the host's read enable. Eight cycles after
the address and read enable signals are set, data comes back on the
memory's data read bus.  From there it gets sliced from 72-bits down
to the data size of the operand in the datapath.  A wire connects the
slice operator with the datapath.

Mapping of the top-level Trident design to the XD1 architecture occurs
in the user\_app.vhd file.  One change to this file is that the status
bus has one more bit. The status bus contains the following
information (in order from lowest to highest bits): register write
enable, memory write enable, memory read enable, and BRAM write
enable.  Other signal mapping is defined by the following:

\begin{verbatim}
      reset_n => reset_n,
      clk       => user_clk,
      i_data  => wide_req,
      i_addr  => s_rt_req.addr,
      i_status => top_status,
      o_reg => top_reg,
      o_mem => top_mem,
      o_bmem => open,
      memi_dr => dr_(i+1),
      memi_dw => dw_(i+1),
      memi_ar => ar_(i+1),
      memi_aw => aw_(i+1),
      memi_re_n => r_n_(i+1),
      memi_we_n => w_n_(i+1)
\end{verbatim}

For examples on signal mapping from the Trident design to the XD1
architecture, see the user\_app.vhd file in the XD1 test directory.

\subsection{Known Bugs/Deficiencies}
\begin{itemize}
\item State machines can exit early for modulo scheduled blocks (?)

\item Something should be done about the parity bits in the data bus
  connections.  I've made the width of the data write buses 72-bits
  (the width of the XD1 memory data buses), while this could be fixed
  outside of the design by either adding parity logic or just
  extending the width of the data.  This will reduce the number of
  bits being OR'ed together inside the top-level and could produce
  faster designs.

\item array designs seem to be very slow (~130MHz).  We may want to
  look at this.
\item It would be more convenient to always have the same memory ports
  on the top-level.  Currently, only the memory ports that are used in
  the design are added.  This makes it difficult because user\_app.vhd
  has to be modified for all designs that use different memory ports.
  Inside the design, the memory ports should just hook up to the host
  logic.

\item Arrays get assigned to memories in a ``non-reproducible''
  manner.  This causes problems when making a fabric.in file.  Always
  remember to check the output of the synthesizer for where the arrays
  have been mapped because often the arrays change to different
  memories.  When this happens, the fabric.in addresses (the portion
  of the addr that selects the memory) have to be changed.  I think
  this could be fixed in the hardware analysis code, but I'm not sure.
  Another possibility is where the synthesizer or AllocateArrays pass
  sets the name of the memory.

\item ConstantMath has a problem which I tried to fix, but didn't have
  the time to thoroughly test.  The problem involved the replacement
  of operands where its defining instruction was reduced.  A HashMap
  was used with the instruction as key and operand as value.  When an
  instruction has more than one operand to replace, there should be
  more than one entry into the HashMap.  By using the instruction as
  key, the entry gets over-written when the second entry is added.
  Therefore, only one operand gets replaced, which leads to incorrect
  CFGs.  I think I've fixed this, but haven't thoroughly tested.
  Also, when ConstantMath tried to reduce Binary instructions, it
  would only look if both operand uses were constant.  This left out
  many potential optimizations that could occur when only one operand
  is a constant. In fact, most of the instructions created in
  ConvertGepInsts were of this type.  I've added code to optimize this
  type of instruction, but its commented-out due to lack of testing.

\end{itemize}