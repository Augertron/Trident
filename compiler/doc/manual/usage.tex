\section{Using the Trident Compiler}
Assuming that everything worked correctly in the previous section, here we explain some
of the options and how to use the Trident compiler.

\subsection{Trident}

The compiler is made up of three separate parts, which are all
integrated in the tcc script.  The first part is the LLVM cfront-end,
which reads the input C file and generates LLVM bytecode.  The next
part of the compiler is llv.  llv takes the generated LLVM bytecode,
optimizes it in LLVM and generates parseable text file, which has the
extension .llv.  This is still a target independent representation.
The .llv file is parsed by the java class fp.Compile, undergoes
further optimization and finally is targeted to a particular floating
point library.  The output result is VHDL.


An example of using tcc:

\begin{verbatim}
tcc -t vhdl if_a.c run
\end{verbatim}

The -t option selects the output type and the next two options if\_a.c
and run are the input file and the function to be compiled.  Trident
will only compile a single function from an input file, the function
cannot have any parameters and cannot return any results.  For
example:

\begin{verbatim}
extern int a,b,c,d;

void run() { 
  if (a < b) {
    a = c;
  } else {
    a = d;
  }
}
\end{verbatim}

The run function does not have any input parameters.  However,
variables that are declared as extern can be used as inputs or outputs
(or both.)  Any variables that are declared as extern will get
registers that can be read and written.  Other variables may not be
able to be seen outside of the datapath.


Options to tcc:

\begin{verbatim}
Usage: tcc [-h] [--llv=/path/to/llv] [compiler-options] input_file function_name

Version 0.5

View compiler options by using the command: 'tcc -h' 
Environment variables LLVMGCCDIR and CLASSPATH must be set correctly
Executables llv and java must be in the user's path
\end{verbatim}


Most options to tcc pass to the backend Trident compiler as does the -h options.  The
backend compiler has the following options:

\begin{verbatim}
Compile [Options] input_file function_name

Short Options:
-a filename     : Filename of architecture (hardware) description
-b              : Generate Testbench -- experimental
-h              : Get help -- good luck!
-l libname      : Specify library, supporting quixilica, aa_fplib, 
-t format       : Target output.  format can be set to "dot" or "vhdl"

Long Options:
--sched=sched_list
        sched_list is comma-separated list which can include
        asap, alap, or fd, and optionally nomod or mod

        It is case insensitive.  You must choose either asap, alap, or force
        directed.  It is default to run modulo scheduling on all loop blocks,
        but to turn that off, you can add "nomod" within the string.  For
        example:

        --sched=fd,nomod

        You can also, specify "mod", but that is the default.

--sched.options=option_list
        option_list is a comma-separated list which can include
        considerpreds (tell schedule not to ignore predicates)
        dontpack (don't pack multiple less than 1-clock tick instructions within a single cycle)
        conserve_area (conserve area by only using as many logic units as 
                necessary so as to not slow down execution of the design)
        fd_maxtrycnt (tell the compiler how many times to attempt force-directed 
                scheduling before giving up)
        ms_maxtrycnt (how many times to attempt modulo scheduling)
        cyclelength (set the length of a clock tick (the default is 1.0))

        The options can be set either by using the long opt, "sched.options" to set several at
        once (like this:

        --sched.options=considerpreds,dontpack,conserve_area

        or they can be set individually by saying:

        --sched.options.considerpreds
\end{verbatim}

Several options may not be very useful and they also may not be very
well tested.  \textbf{-a} is for specifying a different hardware target.
Currently only one hardware target is truly supported -- specifying
other files here is not well tested.  \textbf{-b} generates a skeleton VHDL
testbench for the top-level cell.  It does this by examining the cell,
instancing it and inserting some reset statements.  Treat this option
as experimental -- if it works for you, good -- if it does not, you
were warned.  \textbf{-l} is for specifying additional libraries.  This option
is dynamic and supports the floating point libraries it has been told
about.  aa\_fplib is the library I have supplied. \textbf{-t} specifies the
output format, vhdl or dot.  vhdl produces a vhdl netlist and dot is a
schematic view of that netlist.  Please be aware that hierarchical
netlists quickly exceed dotty's (graphviz) ability to render dot
files.

The long options are described above.  Please note that by default
loops are modulo scheduled and non-loop blocks are scheduled using a
force-directed algorithm.  Other options modify how particulars within
the scheduling algorithms.

In addition to Dot circuit representations, Trident currently creates
some debug information in Dot format.  The
\textit{cfile}\_\textit{function}.dot file is the final optimized
version of the function before synthesis. The synthesized version is
called syn\_\textit{cfile}.dot.  Other dot files represent the data
flow graph of individual hyperblocks.


\subsection{Simulation}

Trident generates a single VHDL file representing the compiled design.
Although, there is just one file, the file contains multiple
heirarchical VHDL Design Units and contains the elements shown in
Figure~\ref{fig:abs_circuit}.  To allow for correct compilation, the
circuit elements are inserted in the VHDL file from the lowest level
of heirarchy (leaves) to the highest level (top).  At the top level,
all of the inputs and outputs are available as registers as well as
\textit{start} and \textit{reset} signals.

\begin{figure}[ht]
\centering
\includegraphics[width=.6\textwidth]{figures/abstract_circuit}
\caption{\label{fig:abs_circuit} Circuit Elements found in the Trident compiler.}
\end{figure}

%
% Compilation and Libraries

Compilation of Trident generated designs usually requires several
support libraries.  It is best to compile the libraries first and then
compile the design to verify that all circuit objects have been
correctly instanced.  A floating-point circuit will require a
floating-point library as well as the trident support library.  The
floating-point library should provide its own means of compilation.
The trident support library has several operations that are not
normally included in most available floating-point libraries.  This
includes casts and a few other functions(e.g., float to int, double to
long, int to float, long to double, float to double, double to float,
fpabs, and fpinv).\footnote{Casts are ugly operations that are easy to
  imply when mixing floats with integer types.}

An example of what compilation may look like:

\begin{verbatim}
> vlib work
> vmap work work
> vmap fplib /my/path/to/fplib/fplib
> vcom -93 example.vhd tb_example.vhd
\end{verbatim}

%
% Test benches.

The compiler does have the ability to generate a skeleton VHDL
testbench for generated designs.  This testbench does not actually
exercise the circuit, but instances it and produces process that
can be used to provide input and test the results.  The generated
testbench saves the user the effort of writing the testbench from
scratch.

%
% expand
The synthesized circuit has a particular model of execution.  The
testbench must follow this model to ensure proper circuit function.
The circuit expects all of the inputs to be available before the start
signal is toggled.  Reset should be toggled before start and will
reset any values written to registers before reset has been toggled.
The circuit design does not eliminate the possibility of modifying the
registers during operation, however, this may result in incorrect
operation.


%\subsection{Final Synthesis}